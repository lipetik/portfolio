% Created 2021-09-24 sex 11:14
% Intended LaTeX compiler: pdflatex
\documentclass[11pt]{article}
\usepackage[utf8]{inputenc}
\usepackage[T1]{fontenc}
\usepackage{graphicx}
\usepackage{grffile}
\usepackage{longtable}
\usepackage{wrapfig}
\usepackage{rotating}
\usepackage[normalem]{ulem}
\usepackage{amsmath}
\usepackage{textcomp}
\usepackage{amssymb}
\usepackage{capt-of}
\usepackage{hyperref}
\author{Filipe Gonçalves Jacinto}
\date{\today}
\title{Decomposição elétrica do TiS3 em condições ambientais}
\hypersetup{
 pdfauthor={Filipe Gonçalves Jacinto},
 pdftitle={Decomposição elétrica do TiS3 em condições ambientais},
 pdfkeywords={},
 pdfsubject={Projeto que mostra que a perda de Enxofre(S) desempenha um papel importante na decomposição elétrica do TiS3 em condições ambientais},
 pdfcreator={Emacs 27.2 (Org mode 9.5)}, 
 pdflang={English}}
\begin{document}

\maketitle
\tableofcontents



\section{Introdução:}
\label{sec:org636aef8}
\subsection{Pontos positivos do TiS3:}
\label{sec:org235eae6}
\begin{itemize}
\item Possível substituto do Silício
\item Bandgap de 1.0 eV
\item Pode ser isolado em monocamada e nanotubos
\end{itemize}
\subsection{Problema do paper: Calcular a decomposição elétrica do TiS3}
\label{sec:orgebbe8f3}
\subsubsection{Resultados:}
\label{sec:orgb670691}
\begin{itemize}
\item A decomposição elétrica é causada tanto por efeito Joule
\item A decomposição elétrica também é causada por formação de vacâncias , mesmo que a energia de ativação da reação seja alta para um único átomo de S
\item O teste com atmosfera rica em Oxigênio mostra que a energia de formação de defeitos cai significativamente quando se retira uma dupla de átomos SO comparado a um átomo de S sozinho
\item DFT + Monte Carlo cinético sugerem que a formação de vacâncias é devida à oxidação do material seguida da dessorção dos átomos de enxofre
\end{itemize}


\section{Estrutura do código:}
\label{sec:orga90337a}

\subsection{Funcoes.py}
\label{sec:org72a91e7}
\subsubsection{Gamma:}
\label{sec:org035e9f3}
\begin{itemize}
\item Faz o cálculo de Gamma[i] cálculo de Gamma
\item \(\Gamma\)
\end{itemize}
\subsubsection{R\textsubscript{sum}:}
\label{sec:org1910e1c}
\begin{itemize}
\item Cálcula R\textsubscript{i}[i]
\item Soma os termos de R\textsubscript{i}[i]
\end{itemize}
\subsubsection{Probable\textsubscript{event}:}
\label{sec:orgcea480e}
\subsubsection{Time\textsubscript{foward}:}
\label{sec:org98e5ec4}
\begin{itemize}
\item Cálcula o time\textsubscript{step} (intervalo do evento) e o adiciona ao contador de tempo(t)
\end{itemize}

\subsection{Main.py:}
\label{sec:org90f0d1c}
\begin{itemize}
\item Este código é dividido em três partes:
\begin{enumerate}
\item A primeira parte declara as variáveis e arrays necessários nesse algoritmo em específicos:

\begin{itemize}
\item \textbf{Variáveis}
\begin{itemize}
\item gamma\textsubscript{zero}
\item temperatura
\item t (tempo)
\end{itemize}
\item \textbf{Arrays}
\begin{itemize}
\item Delta (energias de ativação)
\item N (número de particulas que realizaram o evento )
\end{itemize}
\item \textbf{Empty Arrays}
\begin{itemize}
\item time (armazena os tempos calculados em cada iteração)
\item rate (armazera a soma R\textsubscript{n} ao longo do tempo)
\end{itemize}
\end{itemize}

\item Esta parte é onde está estruturado a ordem do algoritmo utilizando as funções definidas em Funcoes.py

\begin{itemize}
\item O algoritmo segue a seguinte forma:
\begin{enumerate}
\item \textbf{Gamma()} , calcula gamma
\item For in range(numero de iterações necessárias)
\begin{itemize}
\item Lista de eventos dentro do for:
\begin{enumerate}
\item \textbf{R\textsubscript{sum}()}
\item \textbf{Probable Event()}
\item \textbf{Time Foward}
\item Adicionamos o termo no array de \textbf{time}
\item Adicionamos o termo no array de \textbf{rate}
\end{enumerate}
\end{itemize}
\end{enumerate}
\end{itemize}
\item A parte final do algoritmo consiste em plotar/calcular quantidades que ajudem a compreender o problema :
\begin{enumerate}
\item Plot de gráfico utilzando o pacote matplotlib
\end{enumerate}
\end{enumerate}
\end{itemize}
\end{document}
